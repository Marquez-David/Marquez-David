\documentclass[letterpaper,11pt]{article}

\usepackage{latexsym}
\usepackage[empty]{fullpage}
\usepackage{titlesec}
\usepackage{marvosym}
\usepackage[usenames,dvipsnames]{color}
\usepackage{verbatim}
\usepackage{enumitem}
\usepackage[hidelinks]{hyperref}
\usepackage{fancyhdr}
\usepackage[english]{babel}
\usepackage{tabularx}
\input{glyphtounicode}


%----------FONT OPTIONS----------
% sans-serif
% \usepackage[sfdefault]{FiraSans}
% \usepackage[sfdefault]{roboto}
% \usepackage[sfdefault]{noto-sans}
% \usepackage[default]{sourcesanspro}

% serif
% \usepackage{CormorantGaramond}
% \usepackage{charter}


\pagestyle{fancy}
\fancyhf{} % clear all header and footer fields
\fancyfoot{}
\renewcommand{\headrulewidth}{0pt}
\renewcommand{\footrulewidth}{0pt}

% Adjust margins
\addtolength{\oddsidemargin}{-0.5in}
\addtolength{\evensidemargin}{-0.5in}
\addtolength{\textwidth}{1in}
\addtolength{\topmargin}{-.5in}
\addtolength{\textheight}{1.0in}

\urlstyle{same}

\raggedbottom
\raggedright
\setlength{\tabcolsep}{0in}

% Sections formatting
\titleformat{\section}{
  \vspace{-4pt}\scshape\raggedright\large
}{}{0em}{}[\color{black}\titlerule \vspace{-5pt}]

% Ensure that generate pdf is machine readable/ATS parsable
\pdfgentounicode=1

%-------------------------
% Custom commands
\newcommand{\resumeItem}[1]{
  \item\small{
    {#1 \vspace{-2pt}}
  }
}

\newcommand{\resumeSubheading}[4]{
  \vspace{-2pt}\item
    \begin{tabular*}{0.97\textwidth}[t]{l@{\extracolsep{\fill}}r}
      \textbf{#1} & #2 \\
      \textit{\small#3} & \textit{\small #4} \\
    \end{tabular*}\vspace{-7pt}
}

\newcommand{\resumeSubSubheading}[2]{
    \item
    \begin{tabular*}{0.97\textwidth}{l@{\extracolsep{\fill}}r}
      \textit{\small#1} & \textit{\small #2} \\
    \end{tabular*}\vspace{-7pt}
}

\newcommand{\resumeProjectHeading}[2]{
    \item
    \begin{tabular*}{0.97\textwidth}{l@{\extracolsep{\fill}}r}
      \small#1 & #2 \\
    \end{tabular*}\vspace{-7pt}
}

\newcommand{\resumeSubItem}[1]{\resumeItem{#1}\vspace{-4pt}}

\renewcommand\labelitemii{$\vcenter{\hbox{\tiny$\bullet$}}$}

\newcommand{\resumeSubHeadingListStart}{\begin{itemize}[leftmargin=0.15in, label={}]}
\newcommand{\resumeSubHeadingListEnd}{\end{itemize}}
\newcommand{\resumeItemListStart}{\begin{itemize}}
\newcommand{\resumeItemListEnd}{\end{itemize}\vspace{-5pt}}

%-------------------------------------------
%%%%%%  RESUME STARTS HERE  %%%%%%%%%%%%%%%%%%%%%%%%%%%%


\begin{document}

%----------HEADING----------
% \begin{tabular*}{\textwidth}{l@{\extracolsep{\fill}}r}
%   \textbf{\href{http://sourabhbajaj.com/}{\Large Sourabh Bajaj}} & Email : \href{mailto:sourabh@sourabhbajaj.com}{sourabh@sourabhbajaj.com}\\
%   \href{http://sourabhbajaj.com/}{http://www.sourabhbajaj.com} & Mobile : +1-123-456-7890 \\
% \end{tabular*}

\begin{center}
    \textbf{\Huge \scshape David Márquez Mínguez} \\ \vspace{1pt}
    \small +49 01520 6316947 $|$ \href{mailto:x@x.com}{\underline{davidmarquezminguez@gmail.com}} $|$ 
    \href{https://www.linkedin.com/in/davidmarquezminguez/}{\underline{LinkedIn}} $|$
    \href{https://github.com/Marquez-David}{\underline{GitHub}} $|$
    \href{https://marquez-david.vercel.app/}{\underline{Portfolio}}
\end{center}


%-----------EDUCATION-----------
\section{Hochschulbildung}
  \resumeSubHeadingListStart
    \resumeSubheading
      {Universität von Alcala}{Madrid, Spanien}
      {Bachelorstudium Technische Informatik mit Schwerpunkt Informatik}{Aug. 2016 -- Mai 2021}
  \resumeSubHeadingListEnd


%-----------EXPERIENCE-----------
\section{Praktikum}
  \resumeSubHeadingListStart

    \resumeSubheading
      {Cloud Transformation \& Architecture Analyst}{Feb. 2022 -- Sep. 2023}
      {Accenture}{Madrid, Spanien}
      \resumeItemListStart
        \resumeItem{Entwicklung von Funktionalitäten in ServiceNow unter Verwendung von JavaScript.}
        \resumeItem{Entwicklung einer plattformübergreifenden Software, um die Projektanforderungen unter Anwendung benutzerzentrierter Designprinzipien zu erfüllen.}
        \resumeItem{Führen Sie Projekte von der Planungsphase bis zur Auslieferung an den Endkunden mit agilen Methoden.}
      \resumeItemListEnd

    \resumeSubheading
      {Frontend-Entwickler}{März. 2021 -- Jan. 2022}
      {RecomiendApp}{Madrid, Spanien}
      \resumeItemListStart
        \resumeItem{Entwickeln Sie eine Android/IOS-Anwendung mit React Native.}
        \resumeItem{Zusammenarbeit mit dem Designteam, um innovative Lösungen und Verbesserungen bestehender Funktionen zu implementieren.}
        \resumeItem{Verantwortlich für das Produkt-Release-Management und das Multi-Plattform-Publishing.}
    \resumeItemListEnd

  \resumeSubHeadingListEnd


%-----------PROJECTS-----------
\section{Projekte}
    \resumeSubHeadingListStart

      \resumeProjectHeading
          {\textbf{Spotify} $|$ \emph{JavaScript, React Native, REST API, Axios,  ReactQuery, Git}}{Jun. 2023 -- heute}
          \resumeItemListStart
            \resumeItem{Komplette und funktionale Entwicklung einer Spotify-Klon-Anwendung für Android.}
            \resumeItem{Integration mit der Spotify-API zur Anzeige echter Daten.}
            \resumeItem{Code-Optimierung durch Anwendung von Softwaremustern und SOLID-Prinzipien, um eine effiziente Entwicklung zu gewährleisten.}
            \resumeItem{Implementierung von End-to-End-Tests mit Jest, um die Robustheit und Qualität des Codes sicherzustellen.}
          \resumeItemListEnd

         \resumeProjectHeading
          {\textbf{Web-Scraper-Analysator} $|$ \emph{R, Python, Web scraping, LaTex, Git}}{Okt. 2021 -- Feb. 2022}
          \resumeItemListStart
            \resumeItem{Durchführung einer Marktanalyse der am weitesten verbreiteten Web-Scraper.}
            \resumeItem{Umfassende Analyse der Funktionalität und der Algorithmen, die in den Scrapern verwendet werden.}
            \resumeItem{Entwicklung eines Python-Programms zur Bewertung der Wirtschaftlichkeit und Zuverlässigkeit der Scraper.}
            \resumeItem{Vergleich der Ergebnisse durch Diagramme und Bedienfelder.}
          \resumeItemListEnd

       \resumeProjectHeading
          {\textbf{Weitere Projekte} $|$ \emph{Python, JavaScript, ..., HTML, CSS, Git}}{Okt. 2016 -- heute}
          \resumeItemListStart
            \resumeItem{Programmierung von Anwendungen auf CPU und GPU}
            \resumeItem{Entwicklung verschiedener Arten von Projekten im Zusammenhang mit IoT.}
            \resumeItem{Durchführung von Lernmodulen auf Cloud Computing Plattformen wie AWS, Microsoft Azure und Google Cloud.}
          \resumeItemListEnd

    \resumeSubHeadingListEnd
%
%-----------CERTIFICATES-----------
\section{Zertifizierungen}
    \resumeSubHeadingListStart
      \resumeProjectHeading
          {\textbf{Microsoft} $|$ \emph{GitHub Universe Cloud Skills Challenge}}{Nov. 2023}
      \resumeProjectHeading
          {\textbf{Universität Cambridge} $|$ \emph{Erstes Zertifikat in englischer Sprache}}{Feb. 2023}
      \resumeProjectHeading
          {\textbf{ServiceNow} $|$ \emph{Zertifizierter Systemadministrator}}{Okt. 2022}
      \resumeProjectHeading
          {\textbf{ServiceNow} $|$ \emph{Mikrozertifizierung - Flow Designer}}{März. 2022}
    \resumeSubHeadingListEnd

%
%-----------PROGRAMMING SKILLS-----------
\section{Technische Fähigkeiten}
 \begin{itemize}[leftmargin=0.15in, label={}]
    \small{\item{
     \textbf{Programmiersprachen}{: JavaScript, Java, Python, R, Scala, Racket, Prolog, C/C++, Cuda} \\
     \textbf{Frontend-Technologien}{: React, React Native, TypeScript, HTML, CSS} \\
     \textbf{Entwicklertools}{: Git, Docker, Visual Studio Code, NetBeans, Eclipse, Arduino} \\
    }}
 \end{itemize}


%-------------------------------------------
\end{document}